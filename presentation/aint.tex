\documentclass{beamer}
\usetheme{Warsaw}

\title{Amazonian Fabric}
\subtitle{Basic EC2 Instance Management in Python}
\author{@tunixman
http://www.memrise.com/dev/}

\begin{document}

\begin{frame}{Basics}

Why another management framework?

\begin{itemize}

\item Wanted to be up and running quickly.
\item Wasn't familiar with Chef, Puppet, or even cfengine.
\item Was very familiar with sysadmin, Python, boto, and EC2.

\end{itemize}

\end{frame}

\begin{frame}{Starting Instances}

Let's start a backup database server:

\begin{example}
\texttt{\$ start\_instance web web99}
\end{example}

This will:

\begin{itemize}
\item Create a new instance of the ``database'' type.
\item Give it the host name \texttt{db10.memrise.com}.
\item Install MySQL 5.5 from source.
\item Install our \texttt{my.cnf}
\end{itemize}

\end{frame}

\begin{frame}{Instance Types}

  EC2 Instance Types and AMIs are configured using the
  \texttt{instance\_ami\_map} and the \texttt{instance\_types}
  dictionaries in \texttt{start\_instance}. We're using Alestic Ubuntu
  images for our systems, and there are deep assumptions in the
  scripts about that.

\end{frame}

\begin{frame}{\texttt{instance\_type\_map}}
    The mapping between EC2 instance types and our server types
    determines which EC2 instance type is used when
    \texttt{start\_instance} is run.

    \begin{example}
      \begin{center}
        \begin{tabular}[h]{|l|l|}
          \hline
          Host Type & Instance Type \\
          \hline
          default & m1.large \\
          web & c1.medium \\
          database & m2.4xlarge \\
          rabbitmq & m1.small \\
          jenkins & m1.small \\
          \hline
        \end{tabular}
      \end{center}
    \end{example}

\end{frame}

\begin{frame}{\texttt{instance\_ami\_map}}
  The mapping between EC2 instance types and the AMI to build the
  instance with determines which AMI to use for each EC2 instance
  type when \texttt{start\_instance} is run.

    \begin{example}
      \begin{center}
        \begin{tabular}[h]{|l|l|}
          \hline
          Instance Type & Image \\
          \hline
          m1.xlarge & ami\_64bit \\
          m2.4xlarge & ami\_64bit \\
          m1.large & ami\_64bit \\
          c1.medium & ami\_32bit \\
          m1.small & ami\_32bit \\
          \hline
        \end{tabular}
      \end{center}
    \end{example}
\end{frame}

\begin{frame}{Control Server}

  \begin{itemize}
  \item ``Master'' server for running other servers and tracking
    state.
  \item Another system to maintain.
  \item Very good at getting out of sync with running systems.
  \end{itemize}
\end{frame}

\begin{frame}{EC2 Metadata Tags}

  \begin{itemize}
    \item Managed by EC2, accessible by \texttt{boto}
    \item Set by \texttt{setup\_instance}
    \item \texttt{instance\_type} to identify what kind of server it
      is
    \item \texttt{Name} to record host name and which is displayed
      in EC2 console.
    \end{itemize}
\end{frame}

\begin{frame}{Host Names}
  
  \begin{itemize}
  \item set to \texttt{Name} metadata key with the domain suffix
  \item set in \texttt{/etc/hostname}, \texttt{/etc/hosts}, and in
    \texttt{postfix} configuration
  \item CNAME set to AWS host name in Route 53 for DNS to function
  \end{itemize}

\end{frame}

\begin{frame}{Deployment in Parallel}

The \texttt{fabric}\ command for deployments now can:

\begin{itemize}
\item Find the host types that should be deployed to
\item Run common global deployment code once
\item Run common deployment code across all servers in parallel
\item Update the load balancer
\end{itemize}

\end{frame}

\begin{frame}{Deployable Servers}
There are three types of servers we deploy our stack to:

\begin{itemize}
\item Celery
\item Web
\item Solr
\end{itemize}

These are managed with three different \texttt{fabric}\ roles, and
the \texttt{fabric}\ functions for these server types are marked with
their roles.

\end{frame}

\begin{frame}{Parallel Deployment}

\begin{itemize}
\item We use a special wrapper around the \texttt{parallel}\ decorator since
some operating systems don't handle it well. 
\item We deploy to all web servers in parallel where possible.
\item Turns a ten-minute deploy into a 2 minute deploy.
\end{itemize}

\end{frame}

\begin{frame}{Building Role Lists}

\begin{itemize}
\item Create lists of each server type using \texttt{instance\_type}\
  metadata and instance state.
\item Add ssh username and port
\item Cache results to speed repeated calls.
\end{itemize}

\end{frame}

\begin{frame}{Scaling Web Servers}

\begin{itemize}
\item Can't use EC2 Elastic Cloud Service because of metrics.
\item Will often know in advance when load spikes will occur.
\item Want to make adding and removing capacity very easy.
\end{itemize}

\end{frame}

\begin{frame}{Adding and Reducing Capacity}

Adding capacity:

\begin{itemize}
\item Creating new servers with \texttt{setup\_web\_ami}
\item Deploying existing stopped servers with
  \texttt{start\_spare\_web\_servers}
\end{itemize}

Reducing capacity:

\begin{itemize}
\item \texttt{stop\_spare\_web\_servers}
\end{itemize}
\end{frame}

\begin{frame}{Fabric Instance Cache}

But what about the instance cache?

\begin{itemize}
\item It will have stale information after starting or stopping servers.
\item Flushing the cache we punt on and just tell the user to run a
  deploy.
\item This is perfectly fine because of load balancing.
\end{itemize}
\end{frame}

\begin{frame}{Load Balancing}

\begin{itemize}
\item Site stack deployed to all active web servers.
\item New active servers added to load balancer, DNS updated
\item Inactive servers removed from load balancer.
\item When servers inactivated, also removed.
\end{itemize}
\end{frame}

\end{document}

%%% Local Variables: 
%%% mode: latex
%%% TeX-master: t
%%% End: 
